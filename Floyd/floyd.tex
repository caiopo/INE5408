%This is a LaTeX template for homework assignments
\documentclass[a4paper,12pt]{article}
\usepackage[utf8]{inputenc}
\usepackage[brazilian]{babel}
\usepackage{verbatim}
\usepackage{amsmath}
\usepackage[top=1in, bottom=1in]{geometry}

\newcommand{\subscript}[1]{$_{\text{#1}}$}

\title{Análise da complexidade do Algoritmo de Floyd Modificado}
\author{Caio Pereira Oliveira}
\date{10 de maio de 2016}

\begin{document}

\maketitle

\section*{Apresentação do algoritmo}

Considere a matriz de custos D[n,n] definida como no
algoritmo de Floyd. o algoritmo produzirá uma matriz A[n,n], com
comprimentos dos menores caminhos e ainda uma matriz R[n,n] que fornece o vértice k, que é o primeiro vértice a ser visitado no menor caminho de v\subscript{i} até v\subscript{j}.

\begin{verbatim}
INÍCIO
Para i = 1 até n faça
    Para j = 1 ate n faça
        A[i,j] <- D[i,j]
        R[i,j] <- 0
Para i = 1 até n faça
    A[i,j] <- 0
Para k = 1 até n faça
    Para i = 1 até n faça
        Para j = 1 até n faça
            Se A[i,k] + A[k,j] < A[i,j] então faça
                A[i,j] = A[i,k] + A[k,j]
                R[i,j] <- k
FIM
\end{verbatim}

\newpage

\section*{Análise da complexidade}
\subsection*{Primeiro para-faça}
\begin{verbatim}
Para i = 1 até n faça
    Para j = 1 ate n faça
        A[i,j] <- D[i,j]
        R[i,j] <- 0
\end{verbatim}

Analisando de dentro para fora sabemos que:

\begin{itemize}
    \item O custo conteúdo do para-faça interno é $2$, porque são feitas duas atribuições.
    
    \item O para-faça interno possui um custo de $1$ para a inicialização da variável $j$, $n$ para o número de incrementos em $j$ que serão feitos e $n+1$ para os testes. Além disso, todo o seu conteúdo também será repetido $n$ vezes, o que adiciona $2n$ unidades de tempo, resultando em um custo de $1 + n + n+1 + 2n$ que pode ser simplificado para $4n+2$.
    
    \item O para-faça externo possui um custo de $1$ para inicialização da variável $i$, $n$ para o número de incrementos em $i$ que serão feitos e $n+1$ para os testes. Além disso, todo o seu conteúdo também será repetido $n$ vezes, o que adiciona $n*(4n+2)$ unidades de tempo, resultando em um custo de $1 + n + n+1 + n*(4n+2)$ que pode ser simplificado para $4n^2+4n+2$.
\end{itemize}
Logo, o custo total do primeiro para-faça é $4n^2+4n+2$.

\subsection*{Segundo para-faça}
\begin{verbatim}
Para i = 1 até n faça
    A[i,i] <- 0
\end{verbatim}

Analisando de dentro para fora sabemos que:

\begin{itemize}
    \item O custo do conteúdo do para-faça é $1$, porque é feita uma atribuição.
    
    \item O para-faça possui um custo de $1$ para inicialização da variável $i$, $n$ para o número de incrementos em $i$ que serão feitos e $n+1$ para os testes. Além disso, todo o seu conteúdo também será repetido $n$ vezes, o que adiciona $n$ unidades de tempo, resultando em um custo de $1 + n + n+1 + n$ que pode ser simplificado para $3n+2$.
\end{itemize}
Logo, o custo total do segundo para-faça é $3n+2$.

\newpage

\subsection*{Terceiro para-faça}
\begin{verbatim}
Para k = 1 até n faça
    Para i = 1 até n faça
        Para j = 1 até n faça
            Se A[i,k] + A[k,j] < A[i,j] então faça
                A[i,j] = A[i,k] + A[k,j]
                R[i,j] <- k
\end{verbatim}

Analisando de dentro para fora sabemos que:

\begin{itemize}
    \item O custo do conteúdo do teste é $3$, porque são feitas duas atribuições e uma soma.
    
    \item O curso do teste é $2$, porque tem uma adição e uma comparação. Somando com o interior do teste, temos um curso de $5$.
    
    \item O para-faça interno possui um custo de $1$ para inicialização da variável $j$, $n$ para o número de incrementos em $j$ que serão feitos e $n+1$ para os testes. Além disso, todo o seu conteúdo também será repetido $n$ vezes, o que adiciona $5n$ unidades de tempo, resultando em um custo de $1+n+n+1+5n$, que pode ser simplificado para $7n+2$.
    
    \item O para-faça intermediário possui um custo de $1$ para inicialização da variável $i$, $n$ para o número de incrementos em $i$ que serão feitos e $n+1$ para os testes. Além disso, todo o seu conteúdo também será repetido $n$ vezes, o que adiciona $n*(7n+2)$ unidades de tempo, resultando em um custo de $1+n+n+1+n*(7n+2)$, que pode ser simplificado para $7n^2+4n+2$.
    
    \item O para-faça externo possui um custo de $1$ para inicialização da variável $k$, $n$ para o número de incrementos em $k$ que serão feitos e $n+1$ para os testes. Além disso, todo o seu conteúdo também será repetido $n$ vezes, o que adiciona $n*(7n^2+4n+2)$ unidades de tempo, resultando em um custo de $1+n+n+1+n*(7n^2+4n+2)$, que pode ser simplificado para $7n^3 + 4n^2 + 4n + 2$.
\end{itemize}
Logo, o custo total do terceiro para-faça é $7n^3 + 4n^2 + 4n + 2$.

\subsection*{Conclusão}

Após somar o custo dos três para-faça, temos o custo total $7n^3 + 8n^2 + 11n + 6$.\\\\
Analisando o comportamento assintótico desta função, podemos observar que apenar o $n^3$ é predominante para determinar a complexidade do algoritmo.\\\\
Portanto, a complexidade do Algoritmo de Floyd Modificado é $\mathcal{O}(n^3)$


\end{document}
